\author{Andriy Zatserklyaniy, zatserkl@fnal.gov}
\documentclass[english]{article}
\usepackage{babel}
\usepackage{amsmath}
\usepackage[T1]{fontenc}
\usepackage[latin9]{inputenc}   % encoding ISO-8859-9
\usepackage{graphicx}
\usepackage{esint}              % integral symbols
\usepackage{parskip}            % for \smallskip, \medskip and \bigskip
\usepackage{cancel}             % to cancel out variables in text
\usepackage[active]{srcltx}     % enables reverse search by Shift-LeftClick in dvi file
\usepackage{listings}		% to include C++ code

%-- function to scale pictures --%
\makeatletter
\def\ScaleIfNeeded{%
\ifdim\Gin@nat@width>\linewidth
\linewidth
\else
\Gin@nat@width
\fi
}

\begin{document}

% \title{Cherenkov signal from the microchannel-based photomultiplier Photek 240}
\title{Calculations for time-of-flight beamline system for Cherenkov signal from the microchannel-based photomultiplier Photek 240}

\maketitle

\section{Kinematics for TOF calculations}

We would like to develop time-of-flight (TOF) method to separate protons from light particle contaminated the proton beam. 

Because magnetic field selects particles with the same momentum, 
the beam particles have momentum $p$ which is momentum of proton with given kinetic energy $T$. 
Let's $E$ is a total energy of proton with kinetic energy $T$, $E = T+M$, where $M$ is proton's mass. 
\begin{align*}
p^2 & = E^2 - M^2 \\
p^2 & = (T+M)^2 - M^2 \\
& = T^2 + 2MT
\end{align*}
\begin{align}
p = \sqrt{2MT+T^2}
\end{align}
%
NB: the $\sqrt{2MT}$ in the expression for $p$ is non-relativistic momentum. 

Calculate a velocity of particle with mass $m$ and momentum $p$.
%
\begin{align*}
p = \frac{m\beta}{\sqrt{1-\beta^2}} \\ 
p^2(1-\beta^2) = m^2\beta^2 \\ 
p^2 = (p^2+m^2)\beta^2 \\ 
\beta^2 = \frac{p^2}{p^2+m^2}
\end{align*}
%
\begin{align}
\beta = \frac{p}{\sqrt{p^2+m^2}}
\end{align}

\subsection*{Summary of kinematics}

Proton with kinetic energy T has a momentum 
\begin{align*}
p = \sqrt{2MT+T^2}
\end{align*}
Particle with mass $m$ and momentum $p$ has a velocity
\begin{align*}
\beta = \frac{p}{\sqrt{p^2+m^2}}
\end{align*}

\section{The number of photons}

Landau-Lifshitz expression for the slowing-down force which acts on the particle which emits the Cherenkov radiation in the frequency range ($\omega, \omega+d\omega$):

$$
% dF = \frac{e^2}{c^2} \Bigl( 1 - \frac1{\beta^2 n^2} \Bigr) \omega d\omega
dF = \frac{e^2}{c^2} \Bigl( 1 - \frac1{\beta^2 n^2(\omega)} \Bigr) \omega d\omega
$$
where $\beta$ is a particle speed in terms of speed of light, $n(\omega)$ is the refraction index of the medium for the frequency $\omega$. Note that the $\omega$ is a cyclic frequency here, 
$\omega = 2\pi\nu$. 
The force produces a work 
$$
dW = Fdx
$$
on the path $dx$. This work is the energy loss of the particle on the Cherenkov radiation: 
$$
\frac{d^2W}{dx d\omega} = 
\frac{e^2}{c^2} \Bigl( 1 - \frac1{\beta^2 n^2(\omega)} \Bigr) \omega
$$
Because the energy of the photon is 
$$
h\nu = \hbar\omega
$$
the number of photons emitted in the energy range $dW$ is 
$$
dN = \frac1{\hbar\omega}dW
$$
Now, 
$$
\frac{d^2N}{dxd\omega} = \frac{e^2}{\hbar c^2} \Bigl( 1 - \frac1{\beta^2 n^2(\omega)} \Bigr)
$$
Express in terms of the wavelength $\lambda$.
\begin{align*}
& \omega = 2\pi\nu = 2\pi\frac{c}{\lambda} \\
& d\omega = -\frac{2\pi c}{\lambda^2} d\lambda \\
& \frac{d^2N}{dxd\lambda} = \frac{2\pi e^2}{\hbar c} \frac1{\lambda^2}
\Bigl( 1 - \frac1{\beta^2 n^2(\lambda)} \Bigr)\\
\end{align*}
Now 
\begin{align*}
\frac{d^2N}{dxd\lambda} & = - \frac{d^2N}{dxd\omega} \frac{d\omega}{d\lambda} \\
                        & = - \frac{2\pi c}{\lambda^2} \frac{d^2N}{dxd\omega}
\end{align*}
Finally (omit minus sign) 
$$
\frac{d^2N}{dxd\lambda} = \frac{2\pi \alpha}{\lambda^2} 
\Bigl( 1 - \frac1{\beta^2 n^2(\lambda)} \Bigr)
$$
where
$$
\alpha = \frac{e^2}{\hbar c} \approx \frac1{137}
$$

The number of Cherenkov photons in the wavelength range from $\lambda_1$ to $\lambda_2$ emitted by particle on the path $l$ is 
\begin{align*}
N = 2\pi \alpha l
\Bigl( 1 - \frac1{\beta^2 n^2(\lambda)} \Bigr)
\Bigl(\frac1{\lambda_1} - \frac1{\lambda_2} \Bigr)
\end{align*}

\section{Cherenkov radiator: Quarts (fused silica)}

Index of refraction.

Dispersion.
No information.

Light attenuation length.

\section{The number of photoelectrons}

Properties of the Photek 240 microchannel based photomultiplier (MCP PMT). 

% Integral sensitivity: 145 $\alphaA/lm$. 
Integral sensitivity: 145 $\mu A/lm$. 

Spectral sensitivity. 

Quantum efficiency.



\end{document}
