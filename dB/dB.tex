\author{Andriy Zatserklyaniy, zatserkl@fnal.gov}
\documentclass[12pt,english]{article}   % NB: 12pt font
\usepackage{babel}
\usepackage{amsmath}
\usepackage[T1]{fontenc}
\usepackage[latin9]{inputenc}   % encoding ISO-8859-9
\usepackage{graphicx}
\usepackage{esint}              % integral symbols
\usepackage{parskip}            % for \smallskip, \medskip and \bigskip
\usepackage{cancel}             % to cancel out variables in text
\usepackage[active]{srcltx}     % enables reverse search by Shift-LeftClick in dvi file

\usepackage[margin=1in]{geometry}       % "normal" page margins

%-- function to scale pictures --%
\makeatletter
\def\ScaleIfNeeded{%
\ifdim\Gin@nat@width>\linewidth
\linewidth
\else
\Gin@nat@width
\fi
}

\begin{document}

\title{Definition of the dB}

\maketitle

\paragraph{Defenitions}
\ 

A Bel is a measure of the ratio between the two levels. Decibel is 1/10th of the Bel.

A Bel is a decimal logarithm of ratio of the squares of two levels:
$$
R_{Bel} = lg\dfrac{u^2_2}{u^2_1} \\
$$
correspondently, decibel (dB) is
$$
R_{dB} = 10lg\dfrac{u^2_2}{u^2_1} \\
$$
Commonly, this formula has been written as
$$
R_{dB} = 20lg\dfrac{u_2}{u_1} \\
$$
There are two comments to the idea of the decibel:
\begin{itemize}
\item in many cases is convenient to use squares of two levels than the levels itself.
For example, for sound and electromagnetic waves the intensity is equal to square of the amplitude. 
\item human perception works in logarithmic scale: two sounds with amplitude different in 10 times will be percepted by the ears as different in 2 times. 
\end{itemize}

\paragraph{Examples}
\ 
 
20 dB

\begin{align*}
20 = 20lg(u_2/u_1) \\
u_2/u_1 = 10
\end{align*}

10 dB

\begin{align*}
10 = 20lg(u_2/u_1) \\
u_2/u_1 = 10^{0.5} = 3.162 \approx 3.2
\end{align*}

6 dB

\begin{align*}
6 = 20lg(u_2/u_1) \\
u_2/u_1 = 10^{0.3} = 1.995 \approx 2.0
\end{align*}



\end{document}
